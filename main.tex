%%%%%%%%%%%%%%%%%%%%%%%%%%%%%%%%%%%%%%%%%%%%%%%%%%%%%%%%%%%%%%%%%%%%
% File:    Paper.tex based on IEEEtran.bst y IEEEtran.cls          %
% Date     1/21/2020                                               %
% Version: Paper.tex                                               %
% Autor:   Michael Shell                                           %
% Modificated: Ing. Sergio Arriola-Valverde. M.Sc                  %
% paper template for courses                                       %
%******************************************************************%

\documentclass[conference]{IEEEtran}
\usepackage{cite}

\ifCLASSINFOpdf
  % \usepackage[pdftex]{graphicx}
  % declare the path(s) where your graphic files are
  % \graphicspath{{../pdf/}{../jpeg/}}
  % and their extensions so you won't have to specify these with
  % every instance of \includegraphics
  % \DeclareGraphicsExtensions{.pdf,.jpeg,.png}
\else
  % or other class option (dvipsone, dvipdf, if not using dvips). graphicx
  % will default to the driver specified in the system graphics.cfg if no
  % driver is specified.
  % \usepackage[dvips]{graphicx}
  % declare the path(s) where your graphic files are
  % \graphicspath{{../eps/}}
  % and their extensions so you won't have to specify these with
  % every instance of \includegraphics
  % \DeclareGraphicsExtensions{.eps}
\fi
% graphicx was written by David Carlisle and Sebastian Rahtz. It is
% required if you want graphics, photos, etc. graphicx.sty is already
% installed on most LaTeX systems. The latest version and documentation
% can be obtained at: 
% http://www.ctan.org/pkg/graphicx
% Another good source of documentation is "Using Imported Graphics in
% LaTeX2e" by Keith Reckdahl which can be found at:
% http://www.ctan.org/pkg/epslatex
%
% latex, and pdflatex in dvi mode, support graphics in encapsulated
% postscript (.eps) format. pdflatex in pdf mode supports graphics
% in .pdf, .jpeg, .png and .mps (metapost) formats. Users should ensure
% that all non-photo figures use a vector format (.eps, .pdf, .mps) and
% not a bitmapped formats (.jpeg, .png). The IEEE frowns on bitmapped formats
% which can result in "jaggedy"/blurry rendering of lines and letters as
% well as large increases in file sizes.
%
% You can find documentation about the pdfTeX application at:
% http://www.tug.org/applications/pdftex

%%%%%%%%%%%%%%%%%%%%%%%%%%%%%%%%%%%%%%%%%%%%%%%%%%%%%%%%%%%%%%%%%%%%%%%%%%%%%
% Packages used
%%%%%%%%%%%%%%%%%%%%%%%%%%%%%%%%%%%%%%%%%%%%%%%%%%%%%%%%%%%%%%%%%%%%%%%%%%%%%
\usepackage{amsmath}
\usepackage{url}
\usepackage{graphicx}
\usepackage{svg}
\usepackage[hidelinks,bookmarks=false]{hyperref}
\usepackage{scalerel}
\usepackage{tikz}
\usepackage[utf8]{inputenc}
\usepackage[spanish,es-noshorthands]{babel}
\usetikzlibrary{svg.path}
\usepackage{gnuplottex}

%%%%%%%%%%%%%%%%%%%%%%%%%%%%%%%%%%%%%%%%%%%%%%%%%%%%%%%%%%%%%%%%%%%%%%%%%%%%%

%%%%%%%%%%%%%%%%%%%%%%%%%%%%%%%%%%%%%%%%%%%%%%%%%%%%%%%%%%%%%%%%%%%%%%%%%%%%%
% ORCID logo
%%%%%%%%%%%%%%%%%%%%%%%%%%%%%%%%%%%%%%%%%%%%%%%%%%%%%%%%%%%%%%%%%%%%%%%%%%%%%
\definecolor{orcidlogocol}{HTML}{A6CE39}
\tikzset{
  orcidlogo/.pic={
    \fill[orcidlogocol] svg{M256,128c0,70.7-57.3,128-128,128C57.3,256,0,198.7,0,128C0,57.3,57.3,0,128,0C198.7,0,256,57.3,256,128z};
    \fill[white] svg{M86.3,186.2H70.9V79.1h15.4v48.4V186.2z}
                 svg{M108.9,79.1h41.6c39.6,0,57,28.3,57,53.6c0,27.5-21.5,53.6-56.8,53.6h-41.8V79.1z M124.3,172.4h24.5c34.9,0,42.9-26.5,42.9-39.7c0-21.5-13.7-39.7-43.7-39.7h-23.7V172.4z}
                 svg{M88.7,56.8c0,5.5-4.5,10.1-10.1,10.1c-5.6,0-10.1-4.6-10.1-10.1c0-5.6,4.5-10.1,10.1-10.1C84.2,46.7,88.7,51.3,88.7,56.8z};
  }
}

\newcommand\orcidicon[1]{\href{https://orcid.org/#1}{\mbox{\scalerel*{
\begin{tikzpicture}[yscale=-1,transform shape]
\pic{orcidlogo};
\end{tikzpicture}
}{|}}}}
%%%%%%%%%%%%%%%%%%%%%%%%%%%%%%%%%%%%%%%%%%%%%%%%%%%%%%%%%%%%%%%%%%%%%%%%%%%%%

%%%%%%%%%%%%%%%%%%%%%%%%%%%%%%%%%%%%%%%%%%%%%%%%%%%%%%%%%%%%%%%%%%%%%%%%%%%%%
% Rename Keywords name from Palabras Clave
%%%%%%%%%%%%%%%%%%%%%%%%%%%%%%%%%%%%%%%%%%%%%%%%%%%%%%%%%%%%%%%%%%%%%%%%%%%%%
\renewcommand\IEEEkeywordsname{Palabras Clave}
%%%%%%%%%%%%%%%%%%%%%%%%%%%%%%%%%%%%%%%%%%%%%%%%%%%%%%%%%%%%%%%%%%%%%%%%%%%%%

% Document
%%%%%%%%%%%%%%%%%%%%%%%%%%%%%%%%%%%%%%%%%%%%%%%%%%%%%%%%%%%%%%%%%%%%%%%%%%%%%
\begin{document}

\title{Control de Velocidad Angular del Motor CD hps5130}

\author{\IEEEauthorblockN{Fabián Chacón Solano\orcidicon{}\IEEEauthorrefmark{1},
Juan Pérez-Alvarado\IEEEauthorrefmark{1}, 
Andre Marie Ampere\IEEEauthorrefmark{1} y
Ernest Rutherford\IEEEauthorrefmark{1}}
\vspace{2mm}
\IEEEauthorblockA{\IEEEauthorrefmark{1}Escuela de Ingeniería Electrónica,
Instituto Tecnológico de Costa Rica (ITCR), 30101 Cartago, Costa Rica, \\ \{fabichasola, jperez, amampere, erutherford\}@estudiantec.cr}}

% make the title area
\maketitle

\begin{abstract}
En este apartado se resume en cortas palabras de que tratará el informe en términos de metodologías, análisis y resultados importante logrados durante la práctica dirigida. Es importante tomar en cuenta que esta sección deberá tener una extensión entre 100 a 250 palabras como máximo y nunca se utilizarán referencias bibliográficas de ningún tipo. La coherencia y sentido lógico de la redacción es importante para lograr una buena transmisión de las ideas a la hora de redactar documentos con poca extensión y gran volumen de datos y resultados.\\
%\vspace{1mm}
\end{abstract}
\begin{IEEEkeywords}
Motor CD, Primer Orden, Root Locus, PID.
\end{IEEEkeywords}

\IEEEpeerreviewmaketitle

\section{Introducción}

En este informe se realiza la obtención de un modelo de control para un Motor CD con su respectiva verificación. Los objetivos de este laboratorio consisten en modelar analíticamente y empíricamente el Motor CD, diseñar un regulador electrónico PI por ubicación de polos, simular el funcionamiento de la planta, el regulador y el sistema de control usando Matlab, implementar de forma electrónica el regulador para la planta utilizada.

Los alcances de este laboratorio se encuentran dentro de la teoría de sistemas y control automático dada la planificación y desarrollo de un sistema físico. Además, se aprovecha la integración de conocimientos relacionada a análisis y control de sistemas dinámicos, procesamiento digital de señales, electrónica analógica, digital y mixta, junto a normalización técnica. Todo esto resulta en el uso de herramientas modernas de ingeniería para el control de velocidad del motor CD.

\underline{\url{http://www.ie.tec.ac.cr/palvarado/LabCE/lce_guia_informe.pdf}} \cite{Pablo2018}.
\section{Metodología}

El laboratorio consiste en dos partes, inicialmente se utiliza la aplicación Pascal con el fin de obtener el modelo empírico de velocidad del motor hps5130. La adquisición de datos se realiza mediante un experimento dentro de la aplicación descrita y se obtienen los datos del motor, posteriormente almacenados en un archivo con formato CSV.

Los datos obtenidos se procesan con la herramienta ident de matlab, posteriormente se diseña un regulador PI a través de la herramienta sisotool, mediante sisotool se diseña el regulador capaz de estabilizar el sistema en 300ms con un sobreimpulso inferior al 3\% y un error de estado estacionario de 0 ante una entrada escalón con capacidad para eliminar las perturbaciones de entrada o salida a la planta, el regulador propuesto se ajusta a un tiempo de muestreo de 5ms, con el control finalizado se prueba el diseño funcionando con la aplicación Matlab.

La verificación del diseño a nivel computacional se realiza mediante simulink con una entrada escalón y simulaciones, posteriormente se descompone el regulador obtenido a forma paralela con el fin de encontrar las constantes $K_p$ y $K_i$, realizado este proceso se verifica el control de prueba angular constante ante una entrada de prueba escalón de 4krpm y corroborar la eliminación del efecto de las perturbaciones introducidas.

La implementación del diseño se realiza dentro de la aplicación Pascal, se alimenta el kit PSoC 059 y se construye la estructura de control, esta estructura contiene el diseño del PID con la inserción de las constantes $K_p$ y $K_i$, las perturbaciones, la adquisición de datos de la planta y la definición de límites. Con los parámetros seleccionados correctamente se realiza la comprobación física del modelo y se extraen los datos para su análisis.

\section{Análisis de Resultados}
La sección de análisis y conclusión de los resultados obtenidos según el proceso metodológico de medición serán de importancia para lograr conclusiones relevantes y acertadas, es por ello que la redacción y el tiempo verbal deberán ser importantes, es necesario tomar en cuenta que los resultados ya fueron generados, esto como recomendación para la selección del tiempo verbal de la prosa.

Es recomendable por cada resultado mostrado discutir el mismo con fundamentos teóricos y lógicos, es por ello que es importante tomar tiempo y analizar los resultados, ideas sin respaldo teórico ni lógico no son bien vistas. Como apoyo para la discusión es indispensable usar gráficos, figuras, tablas y cuadros los cuales deben de tener conexión con la prosa y además bien legible, de otra forma si no son discutidos y mucho menos legibles no aportan nada al artículo u informe.

Cuando se vayan a realizar comparaciones utilizar niveles de comparación cuantitativos y no cualitativos, esto hace referencia a lo siguiente:

\textbf{``En relación a la figura 1 se ve que cae más brusco en comparación con la figura 2''}, la frase anterior no es clara, pero si se tomará la siguiente redacción es más clara \textbf{``Con base al espectro FM de la figura 1 se cuantificó un piso de ruido de -90 dBm, donde para una frecuencia de 500 kHz la potencia medida es de +5 dBm por encima del nivel de piso de ruido obtenido en la figura 2, es por ello que la desviación FM disminuye en al menos 2\% para los casos analizados.''}

Debido a la naturaleza científico-técnica de los reportes es necesario tener en cuenta la notación de ingeniería adecuada, formato de parámetros y las unidades correctas, es por ello que a continuación se muestran los siguientes casos: $S11$ $\neq$ $S_{11}$, $db$ $\neq$ $dB$, $miliwatt$ $\neq$ $mW$, $microwatt$ $\neq$ $\mu W$, $w$ $\neq$ $\omega$, $kiloohm$ $\neq$ $k \Omega$, entre otras.

En relación a la presentación de ecuaciones es necesario, redactar las ideas de tal manera que la ecuación este autocontenida en el texto y sea de fácil entendimiento para el lector. Para ello tome el siguiente ejemplo de ecuación:

\textbf{``El nivel de detección mínimo (LOD) para un modelo de elevación digital esta descrito por (\ref{LOD})":}
\begin{equation}
    LOD=\delta (z) = \sqrt{(\delta (z)_{DEM_{t\_n}})^2 + (\delta (z)_{DEM_{t\_{n+1}}})^2}
    \label{LOD}
\end{equation}
\textbf{donde $\delta (z)_{DEM_{t\_n}}$ y $\delta (z)_{DEM_{t\_{n+1}}}$ son los valores RMSE obtenidos en el eje $z$ de modelo de elevación digital.}

En relación a las tablas o cuadros, es necesario resumir la información importante, esto con el objetivo de extraer algún comportamiento o tendencia de los datos, sin embargo es importante además utilizar técnicas de estadística descriptiva e inferencial en algunos casos para la discusión de los datos. Al momento de presentar los datos debe ser concisa la prosa y no redundar ni ahondar mucho en la idea. A continuación se muestra un ejemplo de como mostrar los resultados de un cuadro. \textbf{En el cuadro \ref{Noise} se resumen todos los datos experimentales obtenidos para el piso de ruido en dBm para un rango de frecuencias de 500 hasta 1000 kHz en incremento de 100 kHz respectivamente.}
\begin{table}[!htb]
\renewcommand{\arraystretch}{1.3}
\caption{Tendencia del piso de ruido en función de la frecuencia}
\label{Noise}
\centering
\begin{tabular}{c  c}
\hline
\begin{tabular}[|c||c|]{@{}c@{}}$\textbf{Frecuencia}$\\ \textbf{(kHz)} \end{tabular} & \begin{tabular}[|c||c|]{@{}c@{}}$\textbf{Piso de Ruido}$\\ \textbf{(dBm)}\end{tabular}  \\ 
\hline
500 & 1.2239 \\
600 & 1.4576 \\
700 & 1.9860 \\
800 & 1.5680 \\
900 & 1.2370 \\
1000 & 1.5680 \\
\hline
\end{tabular}
\end{table}

\begin{figure}[!ht]
    \centering
    \begin{gnuplot}[terminal=pdf,terminaloptions={font ",20" linewidth 3},scale=0.70]
    set grid
	set ylabel 'Amplitud (V)'
	set xlabel 'Frecuencia rad/s'
	plot sin(x), cos(x), tan(x)
    \end{gnuplot}
    \caption{Relación de tensión-corriente para una bobina, utilizando una señal senoidal con una frecuencia de $f$ = 1 GHz.}
    \label{Methodolody}
\end{figure}

\section{Conclusiones}
 En la sección de conclusiones, es importante responder de manera sistemática los objetivos de la práctica dirigida partiendo desde el general hasta los específicos en prosa nunca en viñetas, para este tipo de documentos como tal. Además de los objetivos, de los resultados experimentales obtenidos y analizados previamente se debe concluir aspectos relevantes que ayuden a dar solidez del artículo o informe, por lo general es necesario ver comparaciones importante a nivel cuantitativo y no cualitativo evitando frases genéricas. Finalmente resultados no obtenidos ni discutidos en el artículo e informe no deberán aparecer en las conclusiones debido a que no tiene sentido alguno discutir de algo que no se llevó acabo.

%%%%%%%%%%%%%%%%%%%%%%%%%%%%%%%%%%%%%%%%%%%%%%%%%%%%%%%%%%%%%%%%%%%%%%%%%%%%%
% Bibliography
\bibliographystyle{IEEEtran}
\bibliography{Paper}
%%%%%%%%%%%%%%%%%%%%%%%%%%%%%%%%%%%%%%%%%%%%%%%%%%%%%%%%%%%%%%%%%%%%%%%%%%%%%
\end{document}


